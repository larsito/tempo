\HeaderA{calc\_sync}{Calculate synchrony indices}{calc.Rul.sync}
\keyword{\textbackslash{}textasciitilde{}kwd1}{calc\_sync}
\keyword{\textbackslash{}textasciitilde{}kwd2}{calc\_sync}
%
\begin{Description}\relax
Function to calculate all or a subset of synchrony indices, including their
three term local quadrat variance (t3) version, and the decomposition of the
variance ratio index into Stotal, Strend, and Sdetrended.
\end{Description}
%
\begin{Usage}
\begin{verbatim}
calc_sync(x, decompose = TRUE, indices = c("eta", "eta_t3", "eta_w",
  "eta_t3_w", "phi", "phi_t3", "varrat", "varrat_t3", "log_varrat",
  "log_varrat_t3"))
\end{verbatim}
\end{Usage}
%
\begin{Arguments}
\begin{ldescription}
\item[\code{x}] temporal community data in a data frame. Species as columns, years
as rows.

\item[\code{decompose}] If TRUE, the outputted synchrony indices will also contain
the total variance ratio (Stotal) and its decomposition into parts
attributable to long term trend (Strend) and actual synchrony (Sdetrended).
See function decompostr.

\item[\code{indices}] A character vector that specifies wich of the synchrony
indices should be calculated. By default, all available indices are
calculated. If only a subset is desired, the names of the functions to
calculate must be provided in the character vector (e.g. c("eta", "phi") to
calculate only standard versions of eta and phi).
\end{ldescription}
\end{Arguments}
%
\begin{Details}\relax
At the moment, the following indices of synchrony are available (in the order they are displayed in the function output, give that all indices haven been calculated):

\code{eta}: The mean correlation of abundances of each species with all other species.

\code{eta\_w}: As eta, but the mean is weighted by the total abundances of species.

\code{eat\_t3}, \code{eta\_t3\_w}: The t3 versions of eta and eta\_w, respectively.

\code{phi}: Phi after Loreau and de Mazancourt (2008).

\code{phi\_t3}: t3 modified version of Phi.

\code{varrat}: variance ratio after Schluter (1984), see also Leps ().

\code{varrat\_t3}: t3 version of the variance ratio.

\code{log\_varrat}: Log transformed variance ratio, see Leps et al. 2018.

\code{log\_varrat\_t3}: t3 version of the log variance ratio.

\code{syn\_total}: The total synchrony, without decomposition (Stotal).

\code{syn\_trend}: The component of total synchrony that is attributable to trend (Strend).

\code{syn\_detrend}: The component of total synchrony that is attributable to year-to-year fluctuations (Sdetrend).
\end{Details}
%
\begin{Value}
A data frame containing all indices that were specified to be
calculated by the indices and decompose arguments.
\end{Value}
%
\begin{Author}\relax
NA\end{Author}
%
\begin{References}\relax
NABoch, M. Fischer, N. Holzel, V. H. Klaus, T. Kleinebecker, M. Tschapka, W.
W. Weisser, and M. M. Gossner. 2016. Land use imperils plant and animal
community stability through changes in asynchrony rather than diversity.
Nat Commun 7:10697.

Gross, K., B. J. Cardinale, J. W. Fox, A. Gonzalez, M. Loreau, H.
W. Polley, P. B. Reich, and J. van Ruijven. 2014. Species richness and the
temporal stability of biomass production: a new analysis of recent
biodiversity experiments. Am Nat 183:1-12.

Leps, J., M. Majekova, A. Vitova, J. Dolezal, and F. de Bello.
2018. Stabilizing effects in temporal fluctuations: management, traits, and
species richness in high-diversity communities. ECOLOGY 99:360-371.

Loreau, M., and C. de Mazancourt. 2008. Species Synchrony and Its
Drivers: Neutral and Nonneutral Community Dynamics in Fluctuating
Environments. The American Naturalist 172:E48-E66.

Schluter, D. 1984. A Variance Test for Detecting Species
Associations, with Some Example Applications. ECOLOGY 65:998-998.
\end{References}
%
\begin{Examples}
\begin{ExampleCode}
##---- Should be DIRECTLY executable !! ----
##-- ==>  Define data, use random,
##--	or do  help(data=index)  for the standard data sets.

## The function is currently defined as
function (x) 
{
  }
\end{ExampleCode}
\end{Examples}
